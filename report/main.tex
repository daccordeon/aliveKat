% aliveKat
\documentclass[aps,pra,superscriptaddress,reprint,nofootinbib]{revtex4-1}
% \documentclass[prb,reprint,nofootinbib]{revtex4-1} 
% \documentclass[pra,superscriptaddress,reprint,nofootinbib]{revtex4-1}
% \documentclass[prb,preprint,letterpaper,noeprint,longbibliography,nodoi,footinbib]{revtex4-1} 



% worry about formatting AFTER the text is written



\usepackage[utf8]{inputenc}
\usepackage{amsmath,amssymb,amsthm}
\usepackage{amsfonts}
\usepackage{graphicx}
\usepackage{float}
\usepackage{mathtools}
% \usepackage[usenames,dvipsnames]{xcolor}	
\usepackage{hyperref}
% \usepackage{siunitx}
\usepackage{textcomp}
\usepackage{subfiles}
\usepackage{comment}
% \usepackage[bottom]{footmisc}

\usepackage{silence}
\WarningFilter{revtex4-1}{Repair the float}

%\bibliographystyle{apsrev4-2}
%\setlength{\parindent}{0pt}



\begin{document}
\title{Optical modelling of advanced gravitational wave detector configurations}

\author{James W. Gardner}
\email{u6069809@anu.edu.au}
\affiliation{%
College of Science, Australian National University, Acton, ACT, 2601, Australia 
}%

\author{Vaishali B. Adya}
% \email{vaishali.adya@anu.edu.au}
\affiliation{%
Centre for Gravitational Astrophysics, The Australian National University, Acton, A.C.T., 2601, Australia
}
\affiliation{%
OzGrav @ ANU, Australian Research Council Centre of Excellence for Gravitational Wave Discovery, Acton, A.C.T., 2601, Australia
}

\author{David McClelland}
% \email{david.mcclelland@anu.edu.au}
\affiliation{%
Centre for Gravitational Astrophysics, The Australian National University, Acton, A.C.T., 2601, Australia
}
\affiliation{%
OzGrav @ ANU, Australian Research Council Centre of Excellence for Gravitational Wave Discovery, Acton, A.C.T., 2601, Australia
}

\author{Daniel Töyra}
% \email{daniel.toyra@anu.edu.au}
\affiliation{%
Centre for Gravitational Astrophysics, The Australian National University, Acton, A.C.T., 2601, Australia
}
\affiliation{%
OzGrav @ ANU, Australian Research Council Centre of Excellence for Gravitational Wave Discovery, Acton, A.C.T., 2601, Australia
}

\date{\today}


%%%%%%%%%%%%%%%%%%%%%%%%%%%%%%%%%%%%%%%%%%
\begin{abstract}
% to-do todo
Abstract: ... 

\end{abstract}

% {
% \let\clearpage\relax
\maketitle
% }
% \tableofcontents

%%%%%%%%%%%%%%%%%%%%%%%%%%%%%%%%%%%%%%%%%%
\section{Introduction}
\label{sec:introduction}

% motivate the need for optical modelling


\section{Finesse}
% section can be quite short
% how it works and why it works, what is new



%%%%%%%%%%%%%%%%%%%%%%%%%%%%%%%%%%%%%%%%%%
\section{Basics of optics for interferometry}

\subsection{Modulation}
% one paragraph

\subsection{Types of cavities}
% one paragraph
% Coupled, overcoupled, undercoupled


\section{Generation and detection of squeezed states}

\subsection{Squeezed states}
% one paragraph

% spectral density

\subsection{Homodyne readout}
% one paragraph



%%%%%%%%%%%%%%%%%%%%%%%%%%%%%%%%%%%%%%%%%%
\section{Gravitational wave detector interferometry}

\subsection{Interferometry}
% one paragraph

\subsection{aLIGO configuration}

\subsection{Signal recycling cavity - short vs.\ long}

\subsection{Internal squeezing}



%%%%%%%%%%%%%%%%%%%%%%%%%%%%%%%%%%%%%%%%%%
\section{Modelling a non-linear element in a cavity}

% threshold results


\subsection{Different definitions of dB}

% amplitude versus power quantities

% finesse does 10*log10(e^r), analytics convention is 20*log10(e^r)



\section{Modelling advanced gravitational wave detector configurations}
% \subsection{Analytics for aLIGO}
% \subsection{Finesse simulation for aLIGO}
% \subsection{Comparison for long SRC with internal squeezing}


\subsection{Reasons for the discrepancy}

% different def of dB gives a factor of 2

% approximations in the analytics



%%%%%%%%%%%%%%%%%%%%%%%%%%%%%%%%%%%%%%%%%%
\section{Future work}
\label{sec:future_work}

% (detuned long SRC ,non degenerate squeezing, limitations of finesse in its current form that can be overcome)


\section{Conclusions}
\label{sec:conclusions}


\begin{acknowledgments}



\end{acknowledgments}


\appendix
% don’t bother with subfiles just yet
% \subfile{appendix.tex}


\nocite{*}
\bibliographystyle{myunsrt}
\bibliography{bib}


\end{document}
