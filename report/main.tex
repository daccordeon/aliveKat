% aliveKat
\documentclass[aps,pra,superscriptaddress,reprint,nofootinbib]{revtex4-1}
% \documentclass[prb,reprint,nofootinbib]{revtex4-1} 
% \documentclass[pra,superscriptaddress,reprint,nofootinbib]{revtex4-1}
% \documentclass[prb,preprint,letterpaper,noeprint,longbibliography,nodoi,footinbib]{revtex4-1} 



% worry about formatting AFTER the text is written



\usepackage[utf8]{inputenc}
\usepackage{amsmath,amssymb,amsthm}
\usepackage{amsfonts}
\usepackage{graphicx}
\usepackage{float}
\usepackage{mathtools}
% \usepackage[usenames,dvipsnames]{xcolor}	
\usepackage{hyperref}
% \usepackage{siunitx}
\usepackage{textcomp}
\usepackage{subfiles}
\usepackage{comment}
% \usepackage[bottom]{footmisc}

\usepackage{silence}
\WarningFilter{revtex4-1}{Repair the float}

%\bibliographystyle{apsrev4-2}
%\setlength{\parindent}{0pt}



\begin{document}
\title{Optical modelling of advanced gravitational wave detector configurations}

\author{James W. Gardner}
\email{u6069809@anu.edu.au}
\affiliation{%
College of Science, Australian National University, Acton, ACT, 2601, Australia 
}%

\author{Vaishali B. Adya}
% \email{vaishali.adya@anu.edu.au}
\affiliation{%
Centre for Gravitational Astrophysics, The Australian National University, Acton, A.C.T., 2601, Australia
}
\affiliation{%
OzGrav @ ANU, Australian Research Council Centre of Excellence for Gravitational Wave Discovery, Acton, A.C.T., 2601, Australia
}

\author{David McClelland}
% \email{david.mcclelland@anu.edu.au}
\affiliation{%
Centre for Gravitational Astrophysics, The Australian National University, Acton, A.C.T., 2601, Australia
}
\affiliation{%
OzGrav @ ANU, Australian Research Council Centre of Excellence for Gravitational Wave Discovery, Acton, A.C.T., 2601, Australia
}

\author{Daniel Töyra}
% \email{daniel.toyra@anu.edu.au}
\affiliation{%
Centre for Gravitational Astrophysics, The Australian National University, Acton, A.C.T., 2601, Australia
}
\affiliation{%
OzGrav @ ANU, Australian Research Council Centre of Excellence for Gravitational Wave Discovery, Acton, A.C.T., 2601, Australia
}

\date{\today}


%%%%%%%%%%%%%%%%%%%%%%%%%%%%%%%%%%%%%%%%%%
\begin{abstract}
% to-do todo
Abstract: ??? 

\end{abstract}

% {
% \let\clearpage\relax
\maketitle
% }
% \tableofcontents

%%%%%%%%%%%%%%%%%%%%%%%%%%%%%%%%%%%%%%%%%%
\section{Introduction}
\label{sec:introduction}

% gw’s and gw detectors
The first detection of gravitational waves in 2015 from the merger of two black holes~\cite{GW150914} opened a new frontier for astronomy. 
Gravitational waves are a prediction of the theory of General Relativity and represent ripples in the ``fabric of space-time'' that stretch and squish the lengths and durations between events. Gravitational wave detectors, such as the Advanced Laser Interferometer Gravitational-wave Observatory (aLIGO~\cite{AdvancedLIGO:2015}), are vast and complex experiments that rely fundamentally on the interference of light to detect minuscule changes in the lengths of the two long arms of an interferometer. Due to the mass scales required, detectable gravitational waves are only generated by the most massive of cosmic sources. With the initial generation of detectors having demonstrated that detection is possible, the natural question now is how to improve upon them to make detectors with higher sensitivity for all kinds of sources.


% motivate the need for optical modelling
Due to the complexity of the detectors, deriving analytic solutions for the expected signal and noise output due to a passing gravitational wave is difficult to do, especially exactly. %, and almost never into a closed form. 
While solutions are known for the current configurations, with each new proposed configuration a new set of non-trivial analytics needs to be done and as the proposals get more exotic the task only grows more difficult.
One solution to this is to look to optical modelling, numerical simulations of an optical system that can deliver fast results and let the researcher quickly ascertain the efficacy of a proposed configuration. 


% what we do
In this report, we demonstrate the use of an existing piece of optical modelling software, Finesse~\cite{finesse}, in a variety of gravitational wave detector configurations and compare against known analytics. In particular, we test the implementation of a non-linear element in Finesse in a simple cavity and then use the element to test proposed detector configurations with internal squeezing.


% report structure
This report is structured as follows.
In Section~\ref{sec:Finesse} we detail Finesse and what it does. In Sections~\ref{sec:basics}~\ref{sec:squeezing} we review the basics of interferometry and squeezing, respectively. In Section~\ref{sec:gwIFO} we detail the aLIGO configuration. In Section~\ref{sec:sqzcavity} we test the implementation of a non-linear element in Finesse against derived analytics. %, outside and inside of a cavity.
In Section~\ref{sec:aLIGOcomparison} we compare a model of the aLIGO configuration in Finesse to known analytics, with and without internal squeezing, for a variety of interferometer parameters.
We suggest areas for future work in Section~\ref{sec:future_work} and draw conclusions in Section~\ref{sec:conclusions}.


\section{Optical modelling - Finesse}
\label{sec:Finesse}
% this section can be quite short

% what does finesse do, how does it do it
Finesse~\cite{finesse} (Frequency domain INterfErometer Simulation SoftwarE) is a piece of optical simulation software designed for modelling interferometer configurations. For a specified configuration of optical components, at each connecting node it calculates the light amplitudes for the carrier, the signal sidebands, and the quantum noise sidebands (see Sections~\ref{sec:basics}). By solving the appropriate matrix equations, it does all of this entirely in the frequency domain. By placing appropriate detectors at the output, Finesse is able to calculate the expected signal and quantum noise transfer functions and so the sensitivity of a proposed detector configuration (see Section~\ref{sec:gwIFO}).
Although not explored here, Finesse also is able to handle the spatial geometry of the configuration by modelling with Gaussian beams.


% what is new
Newly implemented in the development branch of Finesse is a non-linear element component. Unlike the existing squeezer component, this is able to be placed anywhere within the configuration, which allows for internal along with external squeezing. We are interested in both testing if this component is implemented correctly and in using it to examine proposed detector configurations with internal squeezing.


% how we use it
We interact with the simulation through a Python~\cite{python} wrapper called PyKat~\cite{finesse}. All of our working and implementation is documented and available at \url{https://github.com/daccordeon/aliveKat}.


%%%%%%%%%%%%%%%%%%%%%%%%%%%%%%%%%%%%%%%%%%
\section{Basics of optics for interferometry}
\label{sec:basics}

\subsection{Modulation}
% one paragraph

\subsection{Types of cavities}
% one paragraph
% Coupled, overcoupled, undercoupled


\section{Generation and detection of squeezed states}
\label{sec:squeezing}

\subsection{Squeezed states}
% one paragraph

% spectral density

\subsection{Homodyne readout}
% one paragraph



%%%%%%%%%%%%%%%%%%%%%%%%%%%%%%%%%%%%%%%%%%
\section{Gravitational wave detector interferometry}
\label{sec:gwIFO}

\subsection{Interferometry}
% one paragraph

\subsection{aLIGO configuration}

\subsection{Signal recycling cavity - short vs.\ long}

\subsection{Internal squeezing}



%%%%%%%%%%%%%%%%%%%%%%%%%%%%%%%%%%%%%%%%%%
\section{Modelling a non-linear element in a cavity}
\label{sec:sqzcavity}

\subsection{Analytic derivation}



% threshold results


\subsection{Different definitions of dB}

% amplitude versus power quantities

% finesse does 10*log10(e^r), analytics convention is 20*log10(e^r)



\section{Modelling advanced gravitational wave detector configurations}
\label{sec:aLIGOcomparison}
% \subsection{Analytics for aLIGO}
% \subsection{Finesse simulation for aLIGO}
% \subsection{Comparison for long SRC with internal squeezing}


\subsection{Reasons for the discrepancy}

% different def of dB gives a factor of 2

% approximations in the analytics



%%%%%%%%%%%%%%%%%%%%%%%%%%%%%%%%%%%%%%%%%%
\section{Future work}
\label{sec:future_work}

% (detuned long SRC ,non degenerate squeezing, limitations of finesse in its current form that can be overcome)


\section{Conclusions}
\label{sec:conclusions}


\begin{acknowledgments}



\end{acknowledgments}


\appendix
% don’t bother with subfiles just yet
% \subfile{appendix.tex}


\nocite{*}
\bibliographystyle{myunsrt}
\bibliography{bib}


\end{document}
